% Created 2020-11-28 Sat 13:05
% Intended LaTeX compiler: xelatex
\documentclass[presentation,xcolor=table]{beamer}
\PassOptionsToPackage{hyphens}{url}
\usepackage[english]{babel}
\usepackage[T1]{fontenc}
\usepackage[style=apa,natbib=true,hyperref=true,backref=true,maxcitenames=3,url=true,backend=biber,doi=true,isbn=false,eprint=false]{biblatex}
\addbibresource{C:/Users/Noorah/Dropbox/Dissertation/library.bib}
\definecolor{purple_royal}{RGB}{69, 28, 102}
\definecolor{purple_plain}{RGB}{131, 102, 188}
\definecolor{grey}{RGB}{51, 63, 72}
\definecolor{DodgerBlue4}{RGB}{16, 78, 139}
\definecolor{PaleGreen1}{RGB}{154, 255, 154}
\usepackage{amsmath,amsfonts,amssymb,amsthm,enumerate,multirow,array,graphicx,lscape,lastpage,mathabx,csquotes}
\usetheme{CambridgeUS}
\usepackage{fontawesome}
\hypersetup{linktoc = all, colorlinks = true, urlcolor = DodgerBlue4, citecolor = PaleGreen1, linkcolor = black}
\usefonttheme{professionalfonts}
\usepackage{tikz}
\usetikzlibrary{calc}
\usepackage{subfig}
\setbeamercolor{title}{fg=black}
\setbeamercolor{frametitle}{fg=black}
\setbeamercolor{structure}{fg=purple_royal}
\setbeamercolor{section in head/foot}{fg=white, bg=purple_plain}
\setbeamercolor{title in head/foot}{fg=white, bg=purple_royal}
\setbeamercolor{date in head/foot}{fg=grey}
\captionsetup[figure]{labelformat=empty}
\setbeamertemplate{footline}
{
\leavevmode%
\hbox{%
\begin{beamercolorbox}[wd=.75\paperwidth,ht=2.25ex,dp=1ex,center]{title in head/foot}%
\usebeamerfont{author in head/foot}\inserttitle
\end{beamercolorbox}%
%\begin{beamercolorbox}[wd=.3\paperwidth,ht=2.25ex,dp=1ex,center]{section in head/foot}%
%\usebeamerfont{title in head/foot}\insertsection
%\end{beamercolorbox}%
\begin{beamercolorbox}[wd=.25\paperwidth,ht=2.25ex,dp=1ex,center]{date in head/foot}%
\insertframenumber{} / \inserttotalframenumber\hspace*{1ex}
\end{beamercolorbox}}%
\vskip0pt%
}
\subtitle{EmacsConf 2020}
\author{Noorah Alhasan}
\setbeamersize{text margin right=7mm}
\setbeameroption{show notes}
\titlegraphic{\includegraphics[width=0.2\textwidth,height=.25\textheight]{emacsconf.png}
\hspace*{0.5cm}~%
\includegraphics[width=0.2\textwidth,height=.25\textheight]{org-mode-unicorn-logo.png}
\hspace*{0.5cm}~%
\includegraphics[width=0.2\textwidth,height=.25\textheight]{org-roam.png}
\hspace*{0.5cm}~%
\includegraphics[width=0.2\textwidth,height=.25\textheight]{orb.png}}
\usetheme{default}
\author{Noorah Alhasan}
\date{\today}
\title{Org-mode and Org-roam for Scholars and Researchers}
\begin{document}

\maketitle



\begin{frame}[fragile,allowframebreaks]{Introduction}

 \centering
\begin{figure}
\includegraphics[height=0.5\textheight]{meme2.png}
\includegraphics[height=0.5\textheight]{meme1.png}
\caption{The research process: expectation versus reality \textsuperscript{a,b}}
\end{figure}

\raggedright


\scriptsize{\textsuperscript{a} My PhD Quotes. ``Progress is too messy to notice. Don't imagine it like a staircase." \textit{Instagram}, March 3, 2020. Accessed November 25, 2020. \url{https://bit.ly/39loTSf}}

\scriptsize{\textsuperscript{b} My PhD Quotes. ``The most relatable post to describe my current situation." \textit{Instagram}, February 28, 2020. Accessed November 25, 2020. \url{https://bit.ly/3mk2U1K}}

\normalsize

\framebreak

\begin{itemize}
\item Motivation
\begin{itemize}
\item Research is hard
\item Writing is even more difficult
\end{itemize}
\item Goal is to add some \emph{structure} to the \textbf{madness}. (Unfortunately, the madness never stops)
\begin{itemize}
\item \texttt{Org-mode}: planning, time-management, document creation (manuscripts and presentations)
\begin{itemize}
\item \texttt{Org-super-agenda}: customized way of viewing agenda
\item \texttt{Org-sidebar}: buffer-specific agenda/ \texttt{TODO} view.
\item \texttt{Org-roam}: note-taking and general writing, knowledge creation
\item \texttt{Org-roam-bibtex}: reference management, literature review
\item \texttt{Org-roam-server}: visualize your knowledge database
\item \texttt{Org-transclusion}: transclude text from `branch' org files to `main' org file.
\end{itemize}
\end{itemize}
\end{itemize}
\end{frame}

\begin{frame}[label={sec:org2755a23}]{Org-mode modules}
\begin{center}
\includegraphics[width=0.75\textwidth]{org-mode2.png}
\end{center}
\end{frame}



\begin{frame}[label={sec:org8b6026e}]{Planning}
\begin{itemize}
\item Activities
\begin{itemize}
\item Task management (pomodoro method; time-blocking)
\item Time management (appointments; time-blocking)
\end{itemize}
\end{itemize}
\end{frame}

\begin{frame}[fragile,allowframebreaks]{\texttt{TODO}'s and tags}
 \small

\begin{itemize}
\item These are identifiers in an org-file as tasks or reminders.
\item The types of \texttt{TODO}'s can either be set \textbf{globally} in your init file, or they can be \textbf{file/buffer} specific.
\item They are created as a subtree (think `heading'), or in-line (\texttt{'org-inlinetask}).
\item You can assign deadlines, scheduled date and time, active timestamps, and inactive timestamps.
\end{itemize}

\begin{center}
\includegraphics[width=0.7\textwidth]{todo-buffer.png}
\end{center}

\normalsize
\end{frame}

\begin{frame}[label={sec:org920cdb4},fragile]{Org-capture}
 These are customizable org-headings that you can create on-the-go. They can be regular \texttt{TODO}'s or just notes.


\centering
\begin{figure}
\includegraphics[width=0.3\textwidth]{6.png}
\includegraphics[width=0.65\textwidth]{org-capture-config.png}
\end{figure}
\end{frame}


\begin{frame}[fragile,allowframebreaks]{Org-agenda}
 \begin{itemize}
\item Populates all your \texttt{TODO}'s and appointments into a singular view.
\begin{itemize}
\item Default is week-view.
\end{itemize}
\item Using \texttt{org-super-agenda}, I set up my agenda as a daily view with appointments, deadlines, and a habit tracker.
\item \textbf{NOTE}: I am STILL struggling with it, so it's a \textbf{PrOcEsS}.
\end{itemize}

\begin{center}
\includegraphics[width=0.75\textwidth]{org-super-agenda.png}
\end{center}

\centering
\begin{figure}
\includegraphics[width=0.5\textwidth]{org-agenda-timeblocking2.png}
\includegraphics[width=0.45\textwidth]{org-timeblocking.png}
\end{figure}
\end{frame}


\begin{frame}[label={sec:org0e851ba},fragile]{Org-sidebar}
 Another way of accessing your \texttt{TODO}'s that are outside of your agenda. I am using it to keep my project-specific \texttt{TODO}'s in the main project org file.


\begin{center}
\includegraphics[width=0.75\textwidth]{org-sidebar.png}
\end{center}
\end{frame}

\begin{frame}[allowframebreaks]{Literature review (writing + reference management)}
\begin{center}
\includegraphics[width=0.55\textwidth]{lit_review2.png}
\end{center}

\begin{center}
\includegraphics[width=0.9\textwidth]{lit_review.png}
\end{center}

\framebreak

\textbf{Zotero}: open-source reference management software
\begin{itemize}
\item \textbf{BetterBibTex}: automatically creates well-formated bib files and manages reference keys in a systematic way.
\item \textbf{ZotFile}: Efficient plug-in to organize PDFs.
\end{itemize}
\end{frame}

\begin{frame}[fragile,allowframebreaks]{Org-roam}
 A note-taking package that replicates Roam Research which is based on the Zettelkasten method. I use it to build my literature review and I use \texttt{org-roam-server} to visualize my notes into a network.

\begin{itemize}
\item It builds on the strength of \texttt{org-mode}'s hyperlinking properties.
\end{itemize}


\centering
\begin{figure}
\includegraphics[width=0.3\textwidth]{7.png}
\includegraphics[width=0.65\textwidth]{org-roam2.png}
\end{figure}
\end{frame}

\begin{frame}[fragile,allowframebreaks]{Org-roam-bibtex}
 Utilizes a combination of \texttt{org-ref}, \texttt{helm-bibtex}, and \texttt{bibtex-completion} to streamline note-taking workflow with references within the \texttt{org-roam} ecosystem.

\centering
\begin{figure}
\includegraphics[width=0.65\textwidth]{orb-article.png}
\includegraphics[width=0.45\textwidth]{orb-config.png}
\end{figure}

\begin{center}
\includegraphics[width=0.8\textwidth]{orb-book.png}
\end{center}
\end{frame}

\begin{frame}[label={sec:orgd61e046},fragile]{Org-noter}
 \begin{itemize}
\item I use it to annotate PDFs and take notes within the same buffer.
\begin{itemize}
\item Works extremely well with \texttt{PDF-tools}.
\item \texttt{org-noter-create-skeleton}
\end{itemize}
\end{itemize}

\begin{center}
\includegraphics[width=\textwidth]{org-noter.png}
\end{center}
\end{frame}

\begin{frame}[fragile,allowframebreaks]{Org-transclusion}
 \begin{itemize}
\item An effective way of ``copy/pasting'' text from one org file (let's say an org-roam note or a section of your thesis/dissertation) into your main org file.
\item It will export all the transcluded text.
\item Sort of equivalent to ``\texttt{\#+INCLUDE:}''
\end{itemize}

\begin{center}
\includegraphics[width=0.6\textwidth]{org-include.png}
\end{center}

\begin{center}
\includegraphics[width=\textwidth]{org-transclusion.png}
\end{center}
\end{frame}

\begin{frame}[label={sec:org6855e8b},fragile]{Thank you!}
 \begin{block}{Special thanks}
\begin{itemize}
\item \href{https://github.com/jethrokuan}{@jethrokuan} (\texttt{org-roam})
\item \href{https://github.com/zaeph}{@zaeph} (\texttt{org-roam-bibtex}; \texttt{org-roam})
\item \href{https://github.com/myshevchuk}{@myshevchuk} (\texttt{org-roam-bibtex})
\item \href{https://github.com/goktug97}{@goktug97} (\texttt{org-roam-server})
\item \href{https://github.com/nobiot}{@nobiot} (\texttt{org-transclusion})
\item \href{https://github.com/alphapapa}{@alphapapa} (\texttt{org-super-agenda}; \texttt{org-sidebar})
\end{itemize}
\end{block}
\end{frame}
\end{document}
